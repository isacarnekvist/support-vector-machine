\documentclass{article}
\usepackage{fullpage}
\usepackage{pdfpages}
\begin{document}
\section{Linear kernel}

    \begin{verbatim}
    def linear_kernel(x, y):
        return np.dot(x, y) + 1
    \end{verbatim}

    \begin{figure}[ht]
        \centering
        \includegraphics[scale=0.4]{linear}
        \caption{Linear kernel}
    \end{figure}

\section{Radial basis kernel}

    \begin{verbatim}
    def radial_basis_kernel(x, y, sigma=1):
        return np.exp(-np.sum(np.power((x-y), 2))/(2*sigma**2))
    \end{verbatim}

    Larger sigmas seem to make the boundary more ``stiff`` and less prone of over
    fitting.  
    \begin{figure}[ht]
        \centering
        \includegraphics[scale=0.4]{radial_4}
        \includegraphics[scale=0.4]{radial_05}
        \caption{Left: $\sigma = 4$, right $\sigma = 0.5$}
    \end{figure}

    An example where the boundary gets really twisted (figure \ref{fig:radial_overfit}).
    \begin{figure}[ht]
        \centering
        \includegraphics[scale=0.4]{radial_overfit2}
        \caption{Radial basis with a really twisted boundary}
        \label{fig:radial_overfit}
    \end{figure}

\section{Polynomial kernels}

    \begin{verbatim}
    def polynomial_kernel(x, y, p):
        return np.power(linear_kernel(x, y), p)
    \end{verbatim}

    \begin{figure}[ht]
        \centering
        \includegraphics[scale=0.4]{polynomial_3}
        \includegraphics[scale=0.4]{polynomial_5}
        \caption{Polynomial kernels of degree 3 and 4}
    \end{figure}

\section{Code}

    \begin{verbatim}
    from cvxopt.solvers import qp
    from cvxopt.base import matrix, spmatrix
    from itertools import combinations_with_replacement, product
    import numpy as np
    import pylab as pl

    # Kernels
    def linear_kernel(x, y):
        return np.dot(x, y) + 1

    def polynomial_kernel(x, y, p):
        return np.power(linear_kernel(x, y), p)

    def radial_basis_kernel(x, y, sigma=1):
        return np.exp(-np.sum(np.power((x-y), 2))/(2*sigma**2))

    def sigmoid_kernel(x, y, k, delta):
        return np.tanh(k*np.dot(x, y) - delta)

    def build_P_matrix(xs, ts, K):
        """
        :param xs: 2d array where each row is a data point [x, y]
        :param ts: labels corresponding to each row of xs
        :param kernel: handle to a kernel function which only takes two vectors as arguments
                       (rewrite as lambda function if necessary)
        :returns: the matrix t_i t_j K(x_i, x_j)
        """
        points = list(zip(xs, ts))
        n = len(points)
        # List comprehension to get all combinations and apply t_i*t_j*K(x_i, x_j)
        P_ls = np.array([ a[1]*b[1]*K(a[0], b[0]) for a, b in product(points, repeat=2)])
        return matrix(P_ls.reshape((n,n)))

    def qp_args(P):
        n = np.sqrt(len(P))
        G = matrix(-np.eye(n))
        q = matrix(-np.ones(n))
        h = matrix(np.zeros(n))

        return q, G, h

    def indicator(x, xs, labels, alphas, K):
        res = 0
        for i in range(0, len(alphas)):
            #        α       label
            res += alphas[i]*labels[i]*K(xs[i,:], x)

        return res

    # Set up training data
    # The total amount of data points
    n = 50
    np.random.seed(1)
    kernel = lambda x,y: sigmoid_kernel(x, y, 1/2, 0)
    pts1   = np.array([np.random.normal(-1,1,n//2), np.random.normal(-1,1,n//2)]).T
    pts2   = np.array([np.random.normal(1,1,n//2), np.random.normal(1,1,n//2)]).T
    pts    = np.vstack((pts1, pts2))
    labels = [1] * (n//2) + [-1] * (n//2)

    # Set up arguments for cvxopt
    P = build_P_matrix(pts, labels, kernel)
    (q, G, h) = qp_args(P)

    # Optimize
    r = qp(P, q, G, h)
    alphas = r['x']
    alphas = map(lambda x: (x>1e-5)*x, alphas)

    # Calculate decision boundary for plotting
    xr = np.arange(-4, 4, 0.1)
    yr = np.arange(-4, 4, 0.1)
    zvals  = [[indicator([x,y], pts, labels, r['x'], kernel) for y in yr] for x in xr]

    # Plot
    pl.figure()
    pl.plot(pts1[:,1], pts1[:,0], 'bo')
    pl.plot(pts2[:,1], pts2[:,0], 'ro')
    pl.contour(xr, yr, zvals, (-1.0, 0.0, 1.0), linewidths=(1,3,1))
    pl.savefig("polynomial_5.pdf")
    pl.show()
    \end{verbatim}

\end{document}
